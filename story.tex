\documentclass[UTF8,oneside,a4paper]{ctexbook}

% Packages
\usepackage[margin=2.5cm]{geometry}
\usepackage{graphicx}
\usepackage{hyperref}
\usepackage{setspace}
\usepackage{fancyhdr}

% Enable explicit CJK fonts so \emph{} / \textit{} work on Chinese text.
\setCJKmainfont[
    BoldFont = SimHei,
    ItalicFont = SimSun,
    ItalicFeatures = {FakeSlant=0.2}
]{SimSun}

% Document information
\title{Love, Femboy, and The Art of Problem Solving}
\author{Sfever}
\date{\today}

% Page style
\pagestyle{fancy}
\fancyhf{}
\setlength{\headheight}{14pt}
\fancyhead[LO]{\rightmark}

\begin{document}

\maketitle
\tableofcontents

\chapter*{序言}
	一个人最天真的想法,可能就是认为文字拥有传递事情全部真相的能力,这或许有时为真,但在有些时间中,连事件本身都无法带来那种真相。本文所写的便是这样一个事件。参与者之外的所有人都无法体会到事件带来的那种冲击力,甚至不同的参与者有着不同的感受。
	本文既不能也不想传递所有情感,就当作一个笑话吧。纪念一下这段鸡飞狗跳的时光。

	人总是要往前走的,事已至此,先吃饭吧。

% Add your preface here

% Main content
\mainmatter
\chapter*{序章:命运的夜晚}
	几个月之后,当拆鱼刺面对着瓜条时,他会想起那个充满激情的晚上,誓要与柿本广大争高下的那一个时刻。2025年2月14日,在自己家里窝着的长崎素世看到了Ave Mujica的第七集,当时的他还不觉得有什么问题。而落叶木这三个字,似乎也只是停留在生物和地理教科书上的一个名词,作为一个生物系的学生,他和“落叶木”的交集似乎也就只有这么多了。

	二月十五日,他登上了返回学校的飞机。

	二月十六日,他似乎意识到了,Ave Mujica的毁灭,是不可抗拒的力量。

	二月十七日,他在知乎发出了号召,期待着志同道合的伙伴们将Ave Mujica重塑回自己所希望的样子。但当时的他并没有意识到命运似乎向他开了个小小的玩笑,在他面前展开的画卷,属于互联网最黑暗的一面。

	当然远在江苏的另一个人也不会意识到自己将会在未来的腥风血雨中,记录下珍贵的瞬间。



\chapter{战争迷雾}
\section{结合之时}

充满干劲的拆鱼刺,组建了一支充满热情的团队,团队里面不乏出类拔萃的精英,而他们都抱着一个共同的目标:为Ave Mujica的角色们创造一个更好的世界。

	“不应该是这样的”,他想。

	大家都充满干劲,一切都朝着正确的方向发展。就在此时,他遇见了落叶木,一个同样渴望改变许多事情的人。她加入了剧本创作小组,与拆鱼刺一同协作,那段时光格外美好。他们共同讨论剧本中的每一个场景、每一句对白。虽然他只能透过文字来想象对方是怎样的人,但这已经让他心满意足。他完全沉浸在温柔的氛围中,觉得落叶木的一切都无比完美,而自己竟能与这样的人共事,简直如获至宝。多巴胺在体内涌动,他已迫不及待地期待着未来的发展。

他鼓起勇气表白了,也许时机并不完美,但多巴胺早已冲昏了他的头脑。

“管它呢!”拆鱼刺暗自想着,“人生就该抓住机会!”他的冒险最终得到了回报——落叶木温柔地回应了他,让他欣喜若狂。

此刻,拆鱼刺的心情彻底失控,仿佛被幸福的电流击中,整个人都悬浮在梦幻般的喜悦之中。真的成功了吗?每每回想起那天的举动,他都觉得不可思议——怎么会如此冲动,在那个时刻,把深藏心底的情感毫无保留地倾诉出来?但只要一想到落叶木的回应,他便觉得,一切疯狂都值得,即使全世界都不理解也无妨。

他的脑海里不停回放着告白那一刻——没有剧本的修饰,只有最真挚的心声,是勇气和多巴胺交织出的炙热告白。他感觉现实和命运变得轻如鸿毛,只剩下他们之间纯粹而悸动的情感。

一边自嘲自己的“冲动”,一边沉浸在恋爱的甜蜜幻想中,拆鱼刺仿佛踏上了一场荒诞却精彩的冒险。即使结局未卜,他也愿意全力以赴。此刻,他早已不再在意世俗目光,只想和落叶木一起书写属于他们的故事。

对于旁观者来说,这也是一个梦幻的故事,恋爱带来的幸福感不仅影响了拆鱼刺,还同样影响了他的同伴们。为了这个故事能够拥有同样美妙的结局,他们更加努力的工作,每一个细节都精雕细琢,誓要呈现出一个完美的,属于Ave Mujica自己的故事。

\section{热情洋溢}

	喜悦冲昏了拆鱼刺的头脑,他的内心已经完全的属于落叶木了,那个美好,完美,善良又平易近人的女生。他十分笃定未来将是属于他们的,除了这个以外还会有什么其他可能呢?
	
    当然也有过质疑,但是有什么困难是不能克服的呢?抱着如此的心态,时间就这么往前流逝着。对于拆鱼刺来说,学业还是那些学业,课程还是那些课程,令人厌烦的事还是那些令人厌烦的事;与之前不一样的是现在有了落叶木陪他面对这一切。
    
    虽然生活的琐碎和烦恼依然如影随形,但拆鱼刺的心境已悄然改变。他不再独自面对那些枯燥的课程和压力,落叶木的陪伴让他拥有了强大的支撑。两人在平淡中彼此鼓励,一起为梦想奋斗,仿佛任何难题都变得不足为惧。每当疲惫袭来,落叶木的一句温柔安慰便能驱散阴霾;在迎接新挑战时,她的笑容也成为了拆鱼刺继续前行的动力。日常的琐事、学业的压力在他们共同努力下逐渐变得可爱起来。
    
    企划也是在一步一个脚印的稳步前行中,每一刻都有不同的人为企划贡献出属于自己的力量。时间就这么一天一天又一天的过去。

\section{未知全貌}

拆鱼刺的主程坐在上海地铁十一号线上,窗外南翔的风景闪过。他看着面前的手机,上面弹出来了脚本团队的消息。
	
“脚本写完了,您看看什么时候把demo给我们呢”
	
“如果不要素材的话今天吧”,手指在键盘上翻飞。

“这么强?”

“相信我好吧”

列车钻入地下,耳机里的音乐声断断续续。

“我草这地铁里怎么没网啊”

几十分钟后主程拖着疲惫的身躯走进房间,看着大群里落叶木和拆鱼刺的聊天,长叹一口气:“真好啊,我也想有这样的爱情。” 他一边打开VScode一边吐槽:“那帮子文艺青年负责浪漫,而我们负责现实。”

一行行代码在屏幕上出现,一个多小时后一个压缩包出现在了聊天框里。

“说今天就今天”

“老师牛逼!”

看着群里的赞美,心情都好了不少。一天下来饱受摧残的主程就这么昏睡过去了,他梦见了自己之前的一段恋情:同样的美好,看起来也很持久,但最后分崩离析的时候也很迅速。

“草怎么又梦到她了”

\section{悲情前奏}

	美好总是短暂的,当开始时的多巴胺逐渐退去,留下的便是现实的考验。当然,对于那些\emph{真心}相爱着的人来说,这些考验只是会让感情变得更加深刻的宝贵经历而已。拆鱼刺坚信他和落叶木的感情经得起考验,尽管新鲜感已经逐渐消失,但二人之间的情感并没有出现任何裂痕。虽然现实的压力已经逐渐出现在地平线上,但他们时间还多。拆鱼刺很开心的了解到落叶木是需要他的——这并不是一个单方面的关系,这个事实已经足够让拆鱼刺开心了。

    企划也遇到了一些麻烦,虽然说不至于被称之为风雨飘摇,但也是实打实为这对情侣身上增添了新的压力。落叶木的失言让整个企划突然走到了舞台的中央,接受着来自四面八方的审视。当然审视亦或是攻击本身并不是多么值得焦虑的事情。“互联网没有记忆”,拆鱼刺想起了这一句话。只要游戏做出来一切都水到渠成了,他坚信着这一点。到那时候一切争议都会成为历史,唯有未来在等待着他们。

    拆鱼刺对落叶木的情感是真挚且深层的,他不是因为那些浮于表面的东西喜欢上落叶木——金钱,外观,关系。而是那些文字,这段感情是两个相互契合的灵魂的结合,他是如此喜欢落叶木的文字。每一个词,每一句话都投射出那美丽的心灵的一部分。落叶木并不是没有那些肤浅的东西,她有着不错的外表:乌黑如墨的长发从脸庞边垂下,一双炯炯有神的眼睛似乎能看破一切。然而这些在灵魂的契合下都变得无关紧要。

    落叶木也不喜欢那些肤浅的人,在她的互联网上的无数个身份里,唯有背后的灵魂是不变的。拆鱼刺是为数不多能够看破那些伪装,直抵心灵深处的人。

    对于对方灵魂而非外在的爱慕让这段关系变得更加坚实,拆鱼刺和落叶木都能够在这段关系中寻找到安全感——他们现在所急需的一种感觉。刚刚高中毕业的他们十分迷茫。自己以后的道路似乎笼罩在一片迷雾当中,但这段关系就像一盏灯,虽然照亮不了远方但可以照亮近处。至少,他们是这样相信的。

\section{风暴已至}
拆鱼刺的主程坐在电脑面前,享受着来之不易的假期。虽然他的假期不少,但大部分时间都在忙自己的副业和爱好上了。这个假期好不容易所有事情都告一段落了,他打算在游戏上耗费许多的人生。

手机的震动打断了游戏的进程。

    “喂?”

    “哥们快来帮我”

    “帮啥?”

    “落叶木啊”

    “你和落叶木出啥事了?”

    “唉,就是…”

    听着拆鱼刺的长篇大论,主程仿佛感觉到自己的灵魂被吸回了那一天。那时年少无知的他…

    “喂,你还在吗?”

    拆鱼刺的呼唤打断了他的回忆。

    “狗日的小情侣”,他想

    虽然嘴上这么说,但他还是不停给拆鱼刺出主意。拆鱼刺也十分疑惑,为什么一个看似正常的要求会引起对方如此之大的反应呢?难道落叶木就如此抗拒线下的交往,乃至于像他这种已经触及对方内心的人也不能做吗?

    “我不知道,说真的。” 对方也爱莫能助。

    “好吧”,拆鱼刺的心里多了一朵乌云。

    几天以后,这朵乌云像当初物理学大厦的乌云一样,变成了彻底的灾难。拆鱼刺在私聊里收到了一份文件,还有几句话。

    “别试了,哥们。他…根本就是男的”

    他的心情在此刻跌入了谷底。

\section{断壁残垣}
“所以,这就是你知道的一切?”

“是。”

    企划还是那个企划,但拆鱼刺和落叶木早已淡出了。主程还是那个苦逼的主程,只不过让他苦的事早已天翻地覆。




\chapter{知情之人}
\section{错综复杂}
% Your story content here
采访者从主程的聊天框里把注意力移开。

“这样吗?”,他想,“这个落叶木还真是…难以评价”

采访者本人并非不知道落叶木到底是一个什么样的人,但是从整理好的聊天记录和瓜条中了解,和实打实的采访每一个当事人还是不一样的体验。他在聊天记录里面翻找着,寻找下一个采访目标。

“就你了!”,他看到一个熟悉的名字,“户山…又是个少女乐队看上瘾了的家伙。”

“先加个好友吧”

\section{弥天大谎}
采访者找到了那个户山香澄,他是负责运营这个企划的社交账号的那一个人,可能是因为这个身份吧,他对事件里的三方都有不少了解。毕竟,宣传总是最重要的。

“来吧,说说你所知道的”

“我吗?\dots 好吧,其实我知道的并不多,当然如果你是先去找了主程的话那估计比他知道的还是多点。拆鱼刺确实是之后才知道落叶木的真实性别的,但在知道这事之前某人就已经开始作妖了”

“哦?”,采访者发出了好奇的声音。对于这些事他只是有所耳闻,至于具体细节则一概不知。

“怎么说呢\dots 落叶木这人挺能编故事和演戏的,他为了蒙拆鱼刺把啥手段都用上了,甚至生造了一个室友出来。”

“哈?”

“对,然后他编了一通各种玩意出来,连室友强奸+囚禁他都说出来了。”

“不是,这也能信?”

“谁知道呢,反正拆鱼刺没上过几年网,被他蒙的一道一道的。”

“\dots”

“还想听吗?”

“等我消化一下。”

\section{五味杂陈}

采访者翻看着面前的资料,心里无数的想法飘过。

“奈叶?这不是那谁小号吗?”

高町奈叶,户山香澄的小号,曾经借给过正处于风暴中心的落叶木使用过。但在事情平息加上落叶木自己作妖的情况下被收回了。现在的落叶木回到了自己的大号上活跃。

“\dots 然后是现在的落叶木”,采访者随手点开了落叶木的主页,没什么特别的,除了主页里的一行签名。“奈一响停车, 林际叶声如雨。”

“奈\dots 叶\dots”,签名里的异常之处并不难注意到,“原来还想着吗?”

“有意思”

\section{几近癫狂}

采访者回到了和户山香澄的聊天框,“来吧,我准备好了。”

“行,所以那事之后不久他就被踢了出去。”

“所以?”,采访者知道这并不是完全的真相,落叶木在被踢出去之前还干了一件事。他并不知道落叶木具体干了什么,但看起来面前这位也不是很清楚的样子。

“因为他自己乱发言吗,所以就被网暴了。”

“然后就是奈叶的事,对吧?”

“是的。”

“然后之后为什么又收回来呢?”

“他自己拿那个号整烂活,我怕把我拉进去。”

“之后就用回了自己的号?”

“对。”

“之后又找过你?”

“还挺吓人的。那是他自己自立门户之后搞了点东西。”

“童话社对吧?”

“嗯。然后说想要我给他宣发,毕竟我拿着童话社的号嘛。我说不给,他就半夜杀到我家楼下了。”

“嗯\dots 哈?”

\section{风流韵事}

“这都哪门子跟哪门子啊,这落叶木到底想干啥?”,中断了采访的采访者跑去洗了把脸。“感觉我和他们根本不在同一个世界里啊!为什么落叶木都能骗到对象啊!”
采访者摊回了椅子上,怀疑起了自己的人生。如果落叶木都能找到对象的话,那说明他的几率还是蛮大的。但是话又说回来落叶木每天都在与人社交,而自己在干什么?在给他
立传。他身边是各种各样的化妆品,爱慕者和漂亮衣服。而自己身边只有堆成山高的A4纸,还有各种价值不菲的设备。

“往好处想,至少我还算有钱!”,他自言自语道。

但他又想到自己极其失败的感情生活,好像有钱也没啥用,毕竟自己一个人再怎么奢侈好像也就花的出去那么点。

“嘛总之先写落叶木传吧,好歹还有点乐子。”

他翻找着手机里的截图和堆成山一样高的纸堆,“让我看看你到底干了什么,落叶木。”

落叶木和拆鱼刺的聊天记录算不上什么特别难找的东西,毕竟也没有人想着保密这些吧。谁能预料到几个月后的现在居然有人给落叶木写上传记了呢?

“感觉这让我变成了一个更可悲的形象了,怎么回事。”

手指在屏幕上轻点几下,聊天记录便完整的出现在他面前。

“这人\dots 还真是拥有编剧的天赋”,采访者看着一行行出现在自己面前的文字,不由得感叹到,“还有点演技。”

\begin{quotation}
    \itshape{落叶木和拆鱼刺的聊天记录,4月13号,部分内容已删减}

    “老公”

    “我爱你”
    
    “我们分手吧”

    “昨天晚上舍友帮我****”

    “她还弄了我的*”

    “她是女生, 不知道你能不能接受”

    “如果可以的活, 以后你想双飞, 我也能帮你”
\end{quotation}

“傻逼啊我操!”,耳机被摔到了桌上。

\chapter{曲径通幽}
\section{秘密情报}
“至少作为骗子他的人生还是很成功的”,看着面前聊天记录里冲突的几个形象,采访者不由得感叹道。在不同人眼里落叶木拥有着许多不一样的故事,而现在采访者要做的不过是把他们合并起来而已。
虽然说听起来似乎很简单,但这是落叶木的故事。每一个事件都足够击碎几个人的世界观,而现在采访者面对的是这些事件的集合体。

“靠有没有人能给我付精神损失费啊”,他不停的抱怨着。毕竟人一生中也遇不到几次这样的事件,但现在如潮水一般的八卦已经淹没了他的心灵,能够理智的打下每一段文字已经是他尽最大努力之后的
结果了。

“让我看看啊,户山的先放一边,那个还是太超前了。”,他思考着下一个采访对象,“啊,对了,十月。”

童话社目前的剧本组核心成员之一,呢称十月,22岁上下,精神状态正常的人类男性,不出意外的话应该知道不少内幕消息。更重要的是目前还没采访过落叶木的前企划成员,好吧也不是没有,只是主程
那个戾气多到快爆的程序员和落叶木大概没什么交情。

十月倒是很爽快的答应了采访要求。“主要还是太想提供点爆的了”,他说。

“不过在开始之前我有个问题”,对面似乎充满好奇的样子,“你为啥要干这个啊。”

“哦有个人想看”,采访者回答的非常迅速,“以及如果没人把这事记下来的话还是太可惜了。”

“希望你还享受这个过程。”

“一点也不,感觉到了文字的局限性。要不是某人强烈要求早不写了。”

“说的好像落叶木不在用文字一样。”

“以凡人之躯比肩神明还是太困难了。” 随后便是长久的沉默。是啊,落叶木干的事好像不是普通的文字可以表达的。

“\dots 我们先开始吧,祝你好运。”

\section{第二战场}


% Back matter
\backmatter

\chapter{Epilogue}
% Add your epilogue here

\end{document}